\section{8.18}

Let \rs be as random sample from an $N(\theta, \sigma^2)$ population, with $\sigma^2$ known. An LRT of $H_0: \theta = \theta_0$ versus $H_1: \theta \neq \theta_0$ is a test that rejects $H_0$ if

\[
	\frac{|\bar{X} - \theta_0|}{\sigma / \sqrt{n}} > c
\]

\subsection{A}

Find an expression, in terms of standard normal probabilities, for the power function of this test. 

So our goal here is to look at the power function and see if we can't write it in the form of $Z$, a standard normal random variable. This problem is primarily just algebraic rearranging.

What this means is that we want $\bar{X} - \theta$ in the numerator there, not $\theta_0$. 

\begin{align*}
	\beta(\theta) &= P\left( \frac{|\bar{X} - \theta_0|}{\sigma / \sqrt{n}} > c \right) \\ 
	\beta(\theta) &= 1 - P\left( \frac{|\bar{X} - \theta_0|}{\sigma / \sqrt{n}} \leq c \right) & \text{(Flip probability)} \\
	&= 1- P\left(|\bar{X} - \theta_0| \leq \frac{c\sigma}{\sqrt{n}} \right) \\
	&= 1 - P\left( -\frac{c\sigma}{\sqrt{n}} \leq \bar{X} - \theta_0 \leq \frac{c\sigma}{\sqrt{n}} \right) & \text{(Undo Absolute Val)} \\
	&= 1 - P\left( -\frac{c\sigma}{\sqrt{n}} + \theta_0 \leq \bar{X} \leq \frac{c\sigma}{\sqrt{n}} + \theta_0 \right) \\
	&= 1 - P\left( -\frac{c\sigma}{\sqrt{n}} + \theta_0 \leq \bar{X} + \theta - \theta \leq \frac{c\sigma}{\sqrt{n}} + \theta_0 \right) & \text{(Add } \theta - \theta) \\
	&= 1 - P\left( -\frac{c\sigma}{\sqrt{n}} + \theta_0 - \theta \leq \bar{X} - \theta \leq \frac{c\sigma}{\sqrt{n}} + \theta_0 - \theta \right) \\
	&= 1 - P\left( -\frac{c\sigma}{\sqrt{n}} + \theta_0 - \theta \leq (\bar{X} - \theta) \frac{\sigma/\sqrt{n}}{\sigma/\sqrt{n}} \leq \frac{c\sigma}{\sqrt{n}} + \theta_0 - \theta \right)
	& \text{(Mult by 1)} \\
	&= 1 - P\left( 
	-\frac{c(\sigma/\sqrt{n}) + \theta_0 - \theta}{\sigma/\sqrt{n}} 
	\leq \frac{\bar{X} - \theta}{\sigma/\sqrt{n}} 
	\leq -\frac{c(\sigma/\sqrt{n}) + \theta_0 - \theta}{\sigma/\sqrt{n}}  
	\right) \\
	&= 1 - P\left( -c + \frac{\theta_0 - \theta}{\sigma/\sqrt{n}} \leq Z \leq c + \frac{\theta_0 - \theta}{\sigma/\sqrt{n}} \right) \\
	&= 1 + P\left( Z \leq -c + \frac{\theta_0 - \theta}{\sigma/\sqrt{n}} \right) -  P\left( Z \leq c + \frac{\theta_0 - \theta}{\sigma/\sqrt{n}} \right) \\
	&= 1 + F_{Z}\left( -c + \frac{\theta_0 - \theta}{\sigma/\sqrt{n}} \right) - F_{Z}\left(  c + \frac{\theta_0 - \theta}{\sigma/\sqrt{n}} \right)
\end{align*}

Phew. Alright, that's done.

\subsection{B}

The experimenter desires a Type I error probability of $.05$, and a maximum Type II error probability of $.25$ at $\theta = \theta_0 + \sigma$. Find values of $n,c$ that will achieve this.

To find $c$ we will examine the power function given the null hypothesis where $\theta = \theta_0$. This simplifies things down quite a bit.

\begin{align*}
	\beta(\theta) &= 1 + F_Z(-c) - F_Z(c) \\
	0.05 &= 1 + F_Z(-c) - F_Z(c) \\
	0.05 &= 1 + (1 - F_Z(c)) - F_Z(c) \\
	0.05 &= 2 - 2F_Z(c) \\
	-1.95 &= -2F_Z(c) \\
	0.975 &= F_Z(c) \\
	c &= F_Z^{-1}(0.975) \\
	c &= 1.96
\end{align*}

Important note here that we used the inverse norm. We could've done this using calculus but it is the final homework so I used my calculator for this. This gives us the constant we need given a specific probability. 

Now, for $n$ we need to look at the Type II error rate. Here we'll substitute $\theta = \theta_0 + \sigma$. 

Really important to point out here that a Type II error rate of $0.25$ gives us $\beta(\theta \mid H_1) = 1 - P(\text{Type II error}) = 1 - 0.25 = 0.75$.

\begin{align*}
	\beta(\theta_0 + \sigma) &= 1 + F_{Z}\left( -c + \frac{\theta_0 - \theta_0 - \sigma}{\sigma/\sqrt{n}} \right) - F_{Z}\left(  c + \frac{\theta_0 - \theta_0 - \sigma}{\sigma/\sqrt{n}} \right) \\
	&= 1 + F_{Z}\left( -c + \frac{-\sigma}{\sigma/\sqrt{n}} \right) - F_{Z}\left(  c + \frac{-\sigma}{\sigma/\sqrt{n}} \right) \\
	&= 1 + F_{Z}\left( -c - \sqrt{n} \right) - F_{Z}\left(  c - \sqrt{n} \right) \\
	0.75 &= 1 + F_{Z}\left( -1.96 - \sqrt{n} \right) - F_{Z}\left(  1.96 - \sqrt{n} \right) \\
\end{align*}

Note here that $F_Z(-1.96 - \sqrt{n})$ is going to be approximately $0$ for nearly any $n$. For example, $n=10$ already puts this value at $1.5e^{-7}$. So we can simplify this further.

\begin{align*}
	0.75 &=  1 - F_{Z}\left(  1.96 - \sqrt{n} \right) \\
	-.25 &= - F_{Z}\left(  1.96 - \sqrt{n} \right) \\
	.25 &=  F_{Z}\left(  1.96 - \sqrt{n} \right) \\
	F_Z^{-1}(.25) &= 1.96 - \sqrt{n} \\
	-.675 &= 1 - \sqrt{n} \\
	n &= 6.94 \\
	n &\approx 7
\end{align*}

So we would need $c = 1.96$ and $n=7$ to achieve these results.
