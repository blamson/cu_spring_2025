\section{4}

Based on your answer in question 3, explain what small (close to 0) and large (close to 1) values of the likelihood ratio test statistic mean in terms of the probability of parameter values stated in the null and alternative hypotheses. 

\noindent\textbf{Answer:}

Small values of the LRT indicate that the likelihood of the data given the null hypothesis is extremely small in comparison to the likelihood of our data given the alternative. That is, the alternative is far more likely for the data than the null. This would indicate we have a lot of evidence to warrant rejecting the null hypothesis.

Values close to 1 indicate that the likelihoods of the null and alternative are nearly identical, thus implying that the data is just as likely in either case. This would mean low evidence towards rejecting the null. 
