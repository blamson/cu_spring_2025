\section*{1.}

Briefly describe any advantages of using sufficient, minimally sufficient, and/or complete statistics.

Firstly, using sufficient statistics gives us confidence that we haven't lost any crucial information for estimating parameters along the way of creating a statistic. If our statistic isn't sufficient, we know we need more information from the data. When we move down to minimal or complete statistics we have confidence that the statistic we're using is superior to other sufficient statistics. If one statistic we've found for a parameter is complete and another isn't, we have a strong argument for using one over the other. 

So in general, these properties for statistics help us know we have enough information we need for estimation and that we are also are only including what information we need.  

