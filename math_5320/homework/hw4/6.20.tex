\section*{6.20}

For each of the following pdfs let $X_1, \cdots, X_n$ be iid observations. Find a complete sufficient statistic, or show that one does not exist.

\subsection*{A}

\[
	f(x\mid\theta) = \frac{2x}{\theta^2} I_{(0, \theta)}(x) I_{(0, \infty)}(\theta)
\]

Our order of operations here will be to find a sufficient statistic and then verify that it's complete. So, let's start by finding the joint distribution. The big thing to note here is that $x$ is bounded by $\theta$. This will alter the indicator function in the joint pdf once we pass it through the product. 

\begin{align*}
	f(\vec{x} \mid \theta) &= \prod_{i=1}^n  \frac{2x_i}{\theta^2} I_{(0, \theta)}(x_i) I_{(0, \infty)}(\theta) \\
	&= \left( \frac{2x}{\theta^2} \right)^n I_{(0, \theta)}(x_{(n)}) & \text{(Distribute the product)} \\
	&= x^n \left( \frac{2}{\theta^2} \right)^n I_{(0, \theta)}(x_{(n)}) & \text{(Rearrange)}
\end{align*}

Which gives us:

\begin{align*}
	g(T(\vec{x}) \mid \theta) &= \left( \frac{2}{\theta^2} \right)^n I_{(0, \theta)}(x_{(n)}) \\
	h(x) &= x^n \\
	T(\vec{X}) &= x_{(n)}
\end{align*}

Thus, by the factorization theorem, $T(\vec{X}) = x_{(n)}$ is a sufficient statistic for $\theta$.

To verify that this is complete we need to take its expected value. To do that we need a couple things. First, the pdf of the max order statistic. Second, the cdf of the pdf provided. 

The pdf of the order statistics simplifies down greatly for the max. We get:

\begin{align*}
	f_{x_{(n)}}(x) &= \frac{n!}{(n-1)!} f_x(x) (F_x(x))^{n-1} \\
	&= n f_x(x) (F_x(x))^{n-1} 
\end{align*}

Now for the cdf.

\begin{align*}
	F_X(x) &= \int_{t=0}^{t=x} f_X(t) dt \\
	&= \int_{t=0}^{t=x} \frac{dt}{\theta^2} dt \\
	&= \left[ \frac{dt^2}{2\theta^2} \right]_{t=0}^{t=x} \\
	&= \frac{x^2}{\theta^2}
\end{align*}

Plugging this into the pdf for the max gives us:

\begin{align*}
	f_{x_{(n)}}(x) &= n f_x(x) (F_x(x))^{n-1} \\
	&= n \frac{2x}{\theta^2} \left( \frac{x^2}{\theta^2} \right)^{n-1} \\
	&= \frac{2xn}{\theta^2} \left( \frac{x}{\theta} \right)^{2n-2} \\
	&= x^{2n-1}2n\theta^{-2n} \\
	&= \frac{2nx^{2n-1}}{\theta^{2n}}
\end{align*}

Now finally we can look at the expected value of some function of the max.

\begin{align*}
	E[g(x_{(n)})] &= \int_{x=0}^{x=\theta} g(x) \frac{2nx^{2n-1}}{\theta^{2n}} dx \\
	&= \frac{2n}{\theta^{2n}} \int g(x) x^{2n-1} dx
\end{align*}

Some notes. $\frac{2n}{\theta^{2n}}$ will always be nonzero due to the bounds on both $\theta$ and $n$. Similarly, $x$'s bounds also force $x^{2n-1}$ to be non-zero on its support. The only way for the integrand to evaluate to 0 is for $g(x) = 0$ for all $\theta$. Thus, $T(\vec{x}) = X_{(n)}$ is complete.
