\subsection*{D.}

Show that 
\[\frac{p}{q}X \cdot \frac{1}{1+\frac{p}{q}X} \sim Beta(p/2, q/2)\]

For this problem we'll be doing a bit of setup. Instead of going to the $\chi^2$ random variables that make up $X$, I'll be sticking with $X$ for this problem.

Let's let,

\vspace{-4mm}
\begin{align*}
	Y &= \frac{p}{q}X \cdot \frac{1}{1+\frac{p}{q}X} \\
	&=  \frac{p}{q}X \cdot \frac{1}{1+\frac{p}{q}X} \cdot \frac{q}{q} \\
	&= \frac{pX}{q+pX}
\end{align*}

We assign that to $Y$ and multiply by $1$ so we can clean that up a bit. This will make our lives later quite a bit easier. Now let's continue our transformation setup and solve for $X$.

\begin{align*}
	Y &= \frac{pX}{q+pX} \\
	(q+pX)Y &= pX \\
	qY + pXY &= pX \\
	qY &= pX - pXY \\
	qy &= pX(1-Y) \\
	\frac{qY}{1-Y} &= pX \\
	X &= \frac{qY}{p(1-Y)} = g^{-1}(y)
\end{align*}

Next we need $\frac{d}{dy} g^{-1}(y)$. This one takes a second, we'll be leveraging the product rule and powering through the work.

Let's break up $g^{-1}(y)$ into two parts. $f(y) = qy$, $g(y) = (p(1-y))^{-1}$. So, $g^{-1}(y) = f(y)g(y)$.

From this, $f'(y) = q$ and $g'(y) = 1/(p(1-y)^2)$

\begin{align*}
	\frac{d}{dy} g^{-1}(y) &= f(y)g'(y) + f'(y)g(y) \\
	&= \frac{qy}{p(1-y)^2} + \frac{q}{p(1-y)} \\
	&= \frac{qy \cdot p \cdot (1-y) + q \cdot p \cdot (1-y)^2}{p(1-y)^2 p(1-y)} \\
	&= \frac{pq(1-y)(y+1-y)}{p^2(1-y)^3} \\
	&= \frac{q}{p(1-y)^2}
\end{align*}

Okay, time for the change of variable formula now that we have all the pieces we need. Nows a good time to write out our goal. We're looking for the form of a $Beta(p/2, q/2)$ pdf. 

\[
	f_Y(y) = \frac{\Gamma\left( \frac{p+q}{2} \right)}{\Gamma\left( \frac{p}{2} \right) \Gamma\left( \frac{q}{2} \right)}
	y^{\frac{p}{2} - 1} (1-y)^{\frac{q}{2} - 1}
\]

Also I'll go ahead and say that $c = \frac{\Gamma\left( \frac{p+q}{2} \right)}{\Gamma\left( \frac{p}{2} \right) \Gamma\left( \frac{q}{2} \right)}$ because it's really annoying having all of that code floating around all the time. We'll show it the first time then simplify it to keep stuff clean as we progress.

Our primary strategy here is to write out everything, then isolate all the specific variables in hopes that they simplify into the form we need.
\begin{align*}
	f_Y(y) &= f_X(g^{-1}(y)) \cdot \left| \frac{d}{dy} g^{-1}(y) \right| \\ 
	&= \frac{\Gamma\left( \frac{p+q}{2} \right)}{\Gamma\left( \frac{p}{2} \right) \Gamma\left( \frac{q}{2} \right)} 
	\cdot \left( \frac{p}{2} \right)^{\frac{p}{2}} 
	\cdot \left( \frac{qy}{p(1-y)} \right)^{\frac{p}{2} - 1} 
	\cdot \left( 1 + \frac{p}{q}\left( \frac{qy}{p(1-y)} \right) \right)^{-\frac{p+q}{2}}
	\cdot \left| \frac{q}{p(1-y)^2} \right| \\
	&= c \frac{p^{p/2}}{p^{(p/2)-1}p} \cdot \frac{q^{(p/2)-1}q}{q^{p/2}} \cdot y^{\frac{p}{2}-1} \cdot (1-y)^{-((p/2)-1)}(1-y)^2 \cdot \left( 1 + \frac{pqy}{qp(1-y)} \right)^{-\frac{p+q}{2}} \\
	&= c y^{\frac{p}{2}-1} \cdot (1-y)^{-((p/2) + 1)} \cdot \left( 1+\frac{y}{1-y} \right)^{-\frac{p+q}{2}} \\
	&= c y^{\frac{p}{2}-1} \cdot (1-y)^{-((p/2) + 1)} \cdot (1-y)^{\frac{p}{2} + \frac{q}{2}} (1-y+y)^{-\frac{p+q}{2}} \\
	&= c y^{\frac{p}{2}-1} (1-y)^{\frac{q}{2} - 1} \\
	f_Y(y) &= \frac{\Gamma\left( \frac{p+q}{2} \right)}{\Gamma\left( \frac{p}{2} \right) \Gamma\left( \frac{q}{2} \right)} y^{\frac{p}{2}-1} (1-y)^{\frac{q}{2} - 1} 
\end{align*}

Which matches the form we had in our goal. Therefore, 

\[\frac{p}{q}X \cdot \frac{1}{1+\frac{p}{q}X} \sim Beta(p/2, q/2)\]

