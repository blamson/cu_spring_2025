\subsection*{B.}

Derive the mean and variance of $X$.

\subsubsection*{The mean}

Starting with the mean of course. First, let's set our goal. According to the book, the expected value of our $X$ should be:

$$E[X] = \frac{q}{q-2} I_{q > 2}(q)$$

To start, we exploit the fact that $U$ is independent of $V$ here so we can split up the expectations.

\vspace{-0.4cm}
\begin{align*}0
	E[X] &= E\left[ \frac{U/p}{V/q} \right] \\
	&= E[U/p] E\left[ \frac{1}{v/q} \right] \\
	&= pE[U] qE\left[ \frac{1}{v} \right] \\
	&= p\frac{1}{p}qE\left[ \frac{1}{v} \right] \\
	&= qE\left[ \frac{1}{v} \right] \\
\end{align*}

So we're left with just $q$ and $V$. A quick sanity check to the textbook shows that this isn't a red flag, the expected value of an $F$ distribution is only a function of $v_2$. 

So really we just have to evaluate $qE[1/V]$. Easy.

\begin{align*}
	E[X] &= qE[1/V] \\
	&= q \int \frac{1}{v} f_V(v) dv \\
	&= \frac{q}{\Gamma(q/2) 2^{q/2}} \int v^{-1} v^{(q/2)-1} \exp(-v/2) dv \\
	&= \frac{q}{\Gamma(q/2) 2^{q/2}} \int v^{((q/2) -1 )-1} \exp(-v/2) dv \\
\end{align*}

Giving us the kernel of a Gamma random variable with $\alpha = (q/2)-1$ and $\beta = 2$.

Much like in part A we'll be using the inverse of the normalizing constant which is $c^{-1} = \Gamma(\alpha)\beta^{\alpha}$.

We'll also take note of one property of the Gamma function, that is: $\Gamma(\alpha) = (\alpha-1)\Gamma(\alpha - 1)$. 

\begin{align*}
	E[X] &= \frac{q}{\Gamma(q/2) 2^{q/2}} \cdot \Gamma\left( (q/2) - 1 \right) 2^{(q/2)-1}  \\
	&= q \frac{\Gamma((q/2) - 1)}{((q/2) - 1)\Gamma((q/2) - 1))} \cdot \frac{2^{\frac{q}{2} - 1}}{2^{\frac{q}{2}}} \\
	&= \frac{q}{2\left( \frac{q}{2} - 1 \right)} \\
	&= \frac{q}{q-2} I_{q>2}(q)
\end{align*}

Which is the same as the mean stated in the book.

\subsubsection*{The Variance}

For the variance, we'll start with the the form of the variance from the book.

\[
	Var(X) = 2\left( \frac{v_2}{v_2 - 2} \right)^2 \frac{(v_1 + v_2 - 2)}{v_1(v_2-4)}
\]

We'll be using the classic formula for the variance for this problem.

$Var(X) = E[X^2] - E[X]^2$

We already know the latter part of the variance, so our focus now is on $E[X^2]$. 

Using a similar setup from earlier:

\begin{align*}
E[X^2] &= E\left[ \left(\frac{U/p}{V/q}\right)^2 \right] \\
&= E\left[ \frac{U^2}{p^2} \cdot \frac{1}{V^2/q^2} \right] \\
&= \frac{1}{p^2}E[U^2] \cdot q^2E\left[ \frac{1}{V^2} \right] \\
\end{align*}

From here, we focus on $U^2$ for a moment. We know that $E[U^2] = Var(U) + E[U]^2$. The variance of a $\chi^2$ random variable is known, so we'll plug that in.

\begin{align*}
	E[X^2] &= \frac{1}{p^2} \left( Var(U) + E[U^2] \right) \cdot q^2E\left[ \frac{1}{V^2} \right] \\
	&= \frac{1}{p^2} \left( 2p + p^2 \right) \cdot q^2E\left[ \frac{1}{V^2} \right] \\
	&= \frac{1}{p} \left( 2 + p \right) \cdot q^2E\left[ \frac{1}{V^2} \right] \\
\end{align*}

Now, we focus on $1/V^2$. 

\begin{align*}
	E[1/V^2] &= \int \frac{1}{v^2} f_V(v) dv \\
	&= \frac{q}{\Gamma(q/2) 2^{q/2}} \int v^{-2} v^{(q/2)-1} \exp(-v/2) dv \\
	&= \frac{q}{\Gamma(q/2) 2^{q/2}} \int v^{((q/2) - 2 )-1} \exp(-v/2) dv \\
\end{align*}

Giving us the kernel of a Gamma random variable with $\alpha = (q/2)-2$ and $\beta = 2$.

You know the drill by now. We'll plug in the inverse normalizing constant. We're going to do some additional algebra now because it'll come in handy later.

\begin{align*}
	E[1/V^2] &= \frac{1}{\Gamma(q/2) 2^{q/2}} \cdot \Gamma\left( (q/2) - 2 \right) 2^{(q/2)-2}  \\
	&= \frac{\Gamma((q/2) - 2)}{((q/2) - 1)((q/2) - 2)\Gamma((q/2) - 2))} \cdot \frac{2^{\frac{q}{2} - 2}}{2^{\frac{q}{2}}} \\
	&= \frac{1}{4} \cdot \frac{1}{\left( \frac{q}{2} - 1 \right)\left( \frac{q}{2} - 2 \right)} \\
	&= \frac{1}{4} \cdot \frac{1}{\frac{1}{2}(q-2)\frac{1}{2}(q-4)} \\
	&= \frac{1}{(q-2)(q-4)}
\end{align*}

And finally, we return back to our original work and get it into the form we need. This requires a LOT of weird algebra.

Let us reference the form of the variance again.

\[
	Var(X) = 2\left( \frac{v_2}{v_2 - 2} \right)^2 \frac{(v_1 + v_2 - 2)}{v_1(v_2-4)}
\]

Our goal is to slowly split things apart to get all the individual pieces we need and pray it all works out.

\vspace{-3mm}
\begin{align*}
	Var(X) &= E[X^2] - E[X]^2 \\
	Var(X) &= \frac{1}{p}(2+p) \cdot q^2 \frac{1}{(q-2)(q-4)} - \left( \frac{q}{q-2} \right)^2 \\
	&= \frac{(2+p)q^2}{p(q-2)(q-4)} - \frac{q^2}{(q-2)^2} \\
	&= \frac{(2+p)q^2 \cdot (q-2)^2 - q^2(q-2)(q-4) }{p(q-2)(q-4)(q-2)^2} & \text{(Common denominators)} \\
	&= \frac{q^2((2+p) \cdot (q-2)^2 - (q-2)(q-4)) }{(q-2)^2p(q-2)(q-4)} & \text{(Factor out }q^2) \\
	&= \left( \frac{q}{q-2} \right)^2 \cdot  \frac{(q-2)( (2+p) \cdot (q-2) - (q-4)) }{p(q-2)(q-4)} & \text{Factor out } (q-2)) \\ 
	&= \left( \frac{q}{q-2} \right)^2 \cdot  \frac{(2+p) \cdot (q-2) - (q-4) }{p(q-4)} \\ 
	&= \left( \frac{q}{q-2} \right)^2 \cdot  \frac{2q+2p-4 }{p(q-4)} & \text{(Rewrite Numerator)} \\
	&= 2 \cdot \left( \frac{q}{q-2} \right)^2 \cdot  \frac{q+p-2 }{p(q-4)} I_{q > 4}(q) & \text{(Factor out 2)} \\
\end{align*}

And with that we're done!
