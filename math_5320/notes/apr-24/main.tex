%        File: main.tex
%     Created: Thu Sep 05 02:00 PM 2024 M
% Last Change: Thu Sep 05 02:00 PM 2024 M
%
\documentclass[a4paper]{article}
% Head related stuff
\usepackage{fancyhdr}
\pagestyle{fancy}
\fancyhead[R]{MTH 5320 - Statistical Inference}  
\fancyhead[L]{Homework 5} 
\fancyfoot[R]{Brady Lamson}
\fancyfoot[L]{CU Denver}

% Math packages
\usepackage{amsmath}
\usepackage{amsthm}
\usepackage{amsfonts}
\usepackage{setspace}
\usepackage{graphicx}
\graphicspath{ {./images/}}

\setstretch{1.25}

% Plotting packages
%\usepackage{pgfplots}
%\pgfflotsset{width=10cm, compat=1.9}

% Creates quick matrix of any type ----
% NOTE: p for paran, b for brackets, B for curly braces, v for pipes, V for double pipes 
\newcommand{\m}[2]{\begin{#1matrix}#2\end{#1matrix}}

% Creates small p-matrix ----
\newcommand{\psmall}[1]{
	\left(\begin{smallmatrix}
		#1
	\end{smallmatrix} \right)
}

% makes mathbb commands faster to type
\newcommand{\bb}{\mathbb}

\newtheorem*{definition}{Definition}
\newtheorem*{theorem}{Theorem}

\setcounter{secnumdepth}{0}

% Create new command for iid 
\newcommand{\iid}{\stackrel{\text{iid}}{\sim}}

\newcommand{\rs}{$X_1, X_2, ..., X_n$}

\newcommand{\rvmin}[1]{#1_{(1)}}

\newcommand{\rvmax}[1]{#1_{(n)}}
\begin{document}

\section*{N-P Generalization for Exponential Family Proof}

\subsection*{First Bullet}

WLOG, assume $a(\theta), b(\vec{x})$ are positive. Consider the simple hypothesis:

\[H_0: \theta = \theta_0, \;\; H_1: \theta = \theta_1, \;\; \theta_1 > \theta_0\]

The N-P likelihood ratio test is:

\[
	\lambda(\vec{x} \mid \theta_0, \theta_1) = 
	\frac{f(\vec{x}) \theta_0}{f(\vec{x})\theta_1} = 
	\frac{a(\theta_0)}{a(\theta_1)} = 
	\exp\left[ (c(\theta_0) - c(\theta_1))d(\vec{x}) \right]
\]

We will reject $H_0$ when:

\begin{align}
	\frac{a(\theta_0)}{a(\theta_1)} \exp\left[ (c(\theta_0) - c(\theta_1))d(\vec{x}) \right] &\leq k \\
	\implies  \exp\left[ (c(\theta_0) - c(\theta_1))d(\vec{x}) \right] & \leq k_2 \\
	\implies  (c(\theta_0) - c(\theta_1))d(\vec{x}) &\leq k_3 
\end{align}

Since $c(\theta)$ is increasing and $\theta_1 > \theta_0$, $c(\theta_0) - c(\theta_1) < 0$. 

Therefore, we reject when $d(\vec{x}) \geq k_4$. We solve for $k_4$ s.t. $P(d(\vec{x}) \geq k_4) = \alpha$.

Since the dist'n of $d(\vec{x})$ doesn't depend on $\theta$, this will be the UMP test for $\theta > \theta_0$ by the N-P lemma. 

We apply this thought process to the other bullet points found in my stupid iphones photo gallery because I forgot my stupid notebook on my stupid desk at home. Oop.

\section*{Corollary 8.3.B}

Consider a simple test based on the N-P lemma. Suppose that $T(\vec{x})$ is a sufficient statistic for $\theta$ and $g(t \mid \theta_i)$ is the pdf or pmf of $T$ corresponding to $\theta_i$, $i = $ $0$ or $1$. Then, any test based on $T$ with rejection region $S$ is an UMP level $\alpha$ test if it satisfies:

\begin{align*}
	t &\in S \;\; \text{if}\;\; g(t \mid \theta_1) > k g(t \mid \theta_0) \\
	t &\in S^c \;\; \text{if}\;\; g(t \mid \theta_1) < k g(t \mid \theta_0) \\
\end{align*}

for some $k \geq 0$ where $\alpha \geq P(T \in S)$. 

[We have extended the like N-P likelihood ratio test to an equivalent test based on the sufficient statistic $T(\vec{x})$.]

\section*{Definition:}

A family of pdfs or pmfs for a univariate random variable $T$ with real-valued parameter $\theta$ has a monotone likelihood ratio (MLR) if, for every $\theta > \theta_1$, $\frac{g(t \mid \theta_2)}{t \mid \theta_1}$ is a monotone function of $t$ on $\left\{ t: g(t \mid \theta_1) > 0, \; \text{or}\; g(t \mid \theta_2) > 0 \right\}$. 

Note: [$c/0$ is defined as $\infty$ when $c>0$]. 

Any regular exponential family with $g(t\mid \theta) = h(t)c(\theta)\exp(w(\theta)t)$ has a MLR if $w(\theta)$ is a non-decreasing function.

\section{Theorem 8.3.17 (Karlin-Rubin Theorem)}

Consider testing $H_0: \theta \leq \theta_0$ vs. $H_1: \theta > \theta_0$. Suppose that $T$ is a sufficient statistic for $\theta$, and the family of pdfs $\left\{ g(t | \theta): \theta \in \Theta \right\}$ has a MLR. Then, for any $t_0$, the test that rejects $H_0$ iff $T > t_0$ is a UMP level $\alpha$ test, where $\alpha = P(T > t_0)$.

\subsection*{Question}

Whats the difference between the exponential family N-P lemma generalization and the K-R theorem?

K-R theorem doesn't require an exponential family. K-R is a \textbf{level} $\alpha$, not \textbf{size} $\alpha$. 

\end{document}
