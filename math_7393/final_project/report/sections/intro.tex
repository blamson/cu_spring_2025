\section*{Introduction}

\subsection*{Background and Motivation}

Poverty is a multifaceted phenomenon. It is one many of us have experienced. Some of us have escaped and others have perhaps fallen back into it. It's something I have lived through for a good portion of my adulthood. It's a complex topic and thankfully there are a lot of people dedicated to understanding it. The Census Bureau is one of those groups, though, I suppose they have an interest in most everything we as a society get up to. 

Anyway, for some crucial background, how we quantify poverty has changed over time. The Census Bureau's own official poverty measure was crafted back in 1960 and was far simpler than I would ever assume. To quote the SSA, "The poverty thresholds associated with the official measure are the minimum amounts of such income that families of particular sizes and compositions need in order to be considered not poor. When they were developed, the official thresholds represented the cost of a minimum food diet multiplied by 3 (to allow for expenditures on other goods and services)" (\href{https://www.ssa.gov/policy/docs/ssb/v75n3/v75n3p55.html}{Source}). This measure of course resulted in a good degree of criticism and eventually the Census Bureau developed an alternative measure of poverty, the Supplemental Poverty Measure, or the SPM, back in 2011.

This new measure added a lot more nuance and they are still actively developing it to this day. What's useful about this is that, using their data, we get an insight into the variables impacting poverty that they have deemed as meaningful. This gives me a great jumping off point for digging into what affects the odds of a poverty classification. And what things matter is only part of the question. To what degree do those things matter is also very important. I want to identify the demographic variables that are most associated with the SPM poverty measure and to really see what those effects look like from a bayesian perspective. I aim to do that using logistic regression, so we'll be examining which variables influence the odds of a poverty measure from the SPM. 

\subsection*{Data Tables and Sources}

For this project I am pulling data directly from the census bureau. They have a list of enormous 900MB tables dating back to 2009 for the supplemental poverty measure. The table I am using is from their 2023 research table and so all of my analysis is limited to that year specifically. Thankfully this time I do not need to join to many tables; almost everything I need is contained in here. The only other table I need comes from the \textbf{tidycensus} package, which allows for the conversion of state FIPS codes to the corresponding state acronym. All of the datasets I used are linked in the github repository for this project. 

It is worth pointing out again that this data is a simple snapshot from 2023. There are no year-over-year trends to examine or compare against and also no historical grounding in results. As such, it may be risky to generalize these results beyond the scope of 2023. I do believe there is still value at examining these snapshots and seeing what results they produce, however. 