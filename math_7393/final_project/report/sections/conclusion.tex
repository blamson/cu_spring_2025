\section*{Conclusion}

The conclusions from this project have become a little muddier than I had originally thought just a few days ago. This was a project that had gone remarkably smoothly and without issue up until today. To be frank I didn't have much in mind for the conclusion; many of the results were difficult for me to discuss beyond a surface level due to a lack of education in the realm of poverty. However, now that I am actually doing the project I thought I was doing all along we have more to discuss.

First, let's cover some limitations and concerns I have about my variable selection process. This wasn't something I was ever really able to nail down. Until I thought about lasso regression I was kind of just throwing models at the wall and comparing WAIC scores. This is extremely inefficient and non-rigorous and I can't help but feel desperate for a better method. With how computationally complex bayesian modeling is I am cautious over fitting many models. I know that I am definitely not using all of the tools for gauging model performance at my disposal so I wish I was better about that with this project. I want to carry this frustration forward to learn the basic modeling workflow of people who use these types of models professionally. 

I think this frustration also comes from this dataset just being very well thought out. The census bureau has had a long time to craft a clean dataset with the variables they are confident are important. What this means is I rarely had covariates actually straddling zero, so a lot of my model simplifications were done purely off of the desire to try something. What a fascinating problem to run into. 

I am also curious about scaling vs. not scaling the numeric variables. I definitely needed to do something to them, many of their distributions were so awkward that I wasn't quite sure how to handle them. Scaling them seems like a good call, but it hurts interpretation and I'm not sure if there is more I should have done on top of that. Adjusted gross income is a good example of this awkwardness. One may reasonably look at that distribution and find a log transformation appropriate for addressing its extreme right tail. However, that variable contains many zeros and negative values making that no longer feasible. 

As for the results themselves, they actually make me ask more questions. I am now very curious to see what this model trained on different years would look like. How would the intervals for the 2015 model compare to this one? What would happen if I looked at a specific region or state and dug into more granular details? I am now also so extremely curious about the enormous flip the marriage variable made. It makes no sense to me how it could have such a strong increase in the odds of a poverty classification. My only hunch is that it has a relationship I hadn't accounted for in the numeric variables because it flips to an odds decrease when they are removed. If I had found this sooner I would have dug into this very deeply.

Overall though, to actually do some thinking on the topic it is always surreal to see statistics you're used to hearing reflected in the data. People of color are disproportionately affected by so many aspects of poverty and we see that odds increase here in my simple model. Women also experience higher rates of poverty and face many employment struggles in male dominated spaces and we see the odds increase here. Part of me also worries about how the hispanic poverty rate may change in the coming years with frequent ICE raids and mass deporations. Obviously there is more nuance to it and I didn't even include the bulk of the information contained in the original dataset but these thoughts remind of what I was really looking at here.  

I think often it's easy to work on a dataset and forget what it really is. It's easy to get lost in the methods, the numbers and the errors. We shouldn't forget that in this table each row is a person with a life and a family. This isn't just data, these are people and I think that's part of what can be so sobering about seeing reality reflected in the analysis. This dataset is so recent that many of the people classified as in poverty then are probably still there. In a way it almost feels wrong to use their suffering for a simple class project I threw together on a whim. I suppose this is just another reminder that I need to do something good with what I've learned over the years. 